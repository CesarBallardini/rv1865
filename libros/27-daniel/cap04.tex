4:1 Nabucodonosor rey a todos los pueblos naciones, y lenguajes que moran en toda la tierra, paz os sea multiplicada.
4:2 Las señales y milagros que el Alto Dios ha hecho conmigo, conviene que yo las publique.
4:3 ¿Cuán grandes son sus señales, y cuán fuertes sus maravillas? Su reino, reino sempiterno, y su senorío hasta generación y generación.
4:4 Yo Nabucodonosor estaba quieto en mi casa, y florido en mi palacio.
4:5 Ví un sueño que me espantó; y las imaginaciones y visiones de mi cabeza me turbaron en mi cama.
4:6 Por lo cual yo puse mandamiento para hacer venir delante de mí todos los sabios de Babilonia, que me mostrasen la declaración del sueño.
4:7 Y vinieron magos, astrólogos, Caldeos, y adivinos, y dije el sueño delante de ellos: mas nunca me mostraron su declaración:
4:8 Hasta tanto que entró delante de mí Daniel, cuyo nombre es Baltasar, como el nombre de mi Dios, y en el cual hay espíritu de los dioses santos; y dije el sueño delante de él, diciendo:
4:9 Baltasar, príncipe de los magos, yo he entendido que hay en tí espíritu de los dioses santos, y que ningún misterio se te esconde; díme las visiones de mi sueño que he visto, y su declaración.
4:10 Las visiones de mi cabeza en mi cama, eran: Parecíame que veía un árbol en medio de la tierra cuya altura era grande.
4:11 Crecía este árbol, y hacíase fuerte, y su altura llegaba hasta el cielo; y su vista hasta el cabo de toda la tierra.
4:12 Su copa era hermosa, y su fruto en abundancia, y para todos había en él mantenimiento. Debajo de él se ponían a la sombra las bestias del campo, y en sus ramas hacían morada las aves del cielo, y toda carne se mantenía de él.
4:13 Veía en las visiones de mi cabeza en mi cama, y he aquí que un velador y santo descendía del cielo;
4:14 Y clamaba fuertemente, y decía así: Cortád el árbol, y desmochád sus ramas: derribád su copa, y derramád su fruto: váyanse las bestias que están debajo de él, y las aves de sus ramas:
4:15 Mas el tronco de sus raices dejaréis en la tierra, y con atadura de hierro y de metal quede atado en la yerba del campo, y sea mojado con el rocío del cielo, y su vivienda sea con las bestias en la yerba de la tierra:
4:16 Su corazón sea mudado de corazón de hombre, y séale dado corazón de bestia; y pasen sobre él siete tiempos.
4:17 Por sentencia de los veladores se acuerda el negocio, y por dicho de santos la demanda; para que conozcan los vivientes que el Altísimo se enseñorea del reino de los hombres, y a quien él quiere lo dá, y constituye sobre él al más bajo de los hombres.
4:18 Este sueño ví yo el rey Nabucodonosor: mas tú, Baltasar, dirás la declaración de él; porque todos los sabios de mi reino nunca pudieron mostrarme su interpretación: mas tú puedes, porque hay en tí espíritu de los dioses santos.
4:19 Entónces Daniel, cuyo nombre era Baltasar, estuvo callando casi una hora, y sus pensamientos le espantaban. El rey entónces habló, y dijo: Baltasar, el sueño ni su declaración no te espanten. Respondió Baltasar, y dijo: Señor mío, el sueño sea para tus enemigos, y su declaración para los que mal te quieren.
4:20 El árbol que viste, que crecía y se hacía fuerte, y que su altura llegaba hasta el cielo, y su vista por toda la tierra;
4:21 Y su copa era hermosa, y su fruto en abundancia, y que para todos había mantenimiento en él: debajo de él moraban las bestias del campo, y en sus ramas habitaban las aves del cielo:
4:22 Tú mismo eres, o! rey, que creciste, y te hiciste fuerte; y tu grandeza creció, y ha llegado hasta el cielo, y tu señorió hasta el cabo de la tierra.
4:23 Y cuanto a lo que el rey vió, un velador y santo que descendía del cielo, y decía: Cortád el árbol, destruídlo: mas el tronco de sus raices dejaréis en la tierra, y con atadura de hierro y de metal quede atado en la yerba del campo, y sea mojado con el rocío del cielo, y su vivienda sea con las bestias del campo, hasta que pasen sobre él siete tiempos:
4:24 Esta es la declaración, o! rey, y la sentencia del Altísimo, que ha venido sobre el rey mi Señor.
4:25 Que te echarán de entre los hombres, y con las bestias del campo será tu morada, y con yerba del campo te apacentarán como a los bueyes, y con rocío del cielo serás teñido; y siete tiempos pasarán sobre tí, hasta que entiendas que el Altísimo se enseñoréa del reino de los hombres, y que a quien él quisiere, lo dará.
4:26 Y lo que dijeron, que dejasen en la tierra el tronco de las raices del mismo árbol: tu reino se te quedará firme, para que entiendas que el señorío es en los cielos.
4:27 Por tanto, o! rey, aprueba mi consejo, y redime tus pecados con justicia, y tus iniquidades con misericordias de los pobres: he aquí la medicina de tu pecado.
4:28 Todo vino sobre el rey Nabucodonosor.
4:29 Al cabo de doce meses andándose paseando sobre el palacio del reino de Babilonia,
4:30 Habló el rey, y dijo: ¿No es esta la gran Babilonia, que yo edifiqué para casa del reino, con la fuerza de mi fortaleza, y para gloria de mi grandeza?
4:31 Aun estaba la palabra en la boca del rey, cuando cae una voz del cielo: A tí dicen, rey Nabucodonosor: El reino es transpasado de tí:
4:32 Y de entre los hombres te echan, y con las bestias del campo será tu morada, y como a los bueyes te apacentarán; y siete tiempos pasaran sobre tí, hasta que conozcas que el Altísimo se enseñoréa en el reino de los hombres, y a quien él quisiere lo dará.
4:33 En la misma hora se cumplió la palabra sobre Nabucodonosor, y fué echado de entre los hombres, y comía yerba como los bueyes, y su cuerpo se teñía con el rocío del cielo, hasta que su pelo creció como de águila, y sus uñas como de aves.
4:34 Mas al fin del tiempo, yo Nabucodonosor, alcé mis ojos al cielo, y mi sentido me fué vuelto, y bendije al Altísimo, y alabé, y glorifiqué al que vive para siempre; porque su señorío es sempiterno, y su reino por todas las edades:
4:35 Y todos los moradores de la tierra por nada son contados; y en el ejército del cielo, y en los moradores de la tierra hace según su voluntad, ni hay quien lo estorbe con su mano, y le diga: ¿Qué haces?
4:36 En el mismo tiempo mi sentido me fué vuelto, y torné a la majestad de mi reino: mi hermosura y mi grandeza volvió sobre mí; y mis gobernadores y mis grandes me buscaron, y fuí restituido en mi reino, y mayor grandeza me fué añadida.
4:37 Ahora yo Nabucodonosor alabo, engrandezco, y glorifíco al Rey del cielo, porque todas sus obras son verdad, y sus caminos juicio; y a los que andan con soberbia puede humillar.
