% Chapitre 2
\bchapter
%\lehead[\hspace{-3em}\Large\pagemark\quad\normalfont\headerfont Creation de l'homme.\quad Genese.]{}
\pagestyle{scrplain}

%\begin{chaptercomment}
%\end{chaptercomment}

\vspace{\baselineskip}

\bversenonum \lettrine[lines=3,loversize=-0.2,lraise=0.2]{Y}{}en el segundo año del reino de Nabucodonosor, soñó Nabucodonosor sueños, y su espíritu se quebrantó, y su sueño se huyó de él.

\bverse Y mandó el rey llamar magos, astrólogos, y encantadores, y Caldeos, para que enseñasen al rey sus sueños: los cuales vinieron, y se presentaron delante del rey.
\bverse Y el rey les dijo: He soñado un sueño, y mi espíritu se ha quebrantado por saber el sueño.
\bverse Y los Caldeos hablaron al rey en Siriaco: Rey, para siempre vive: Dí el sueño a tus siervos, y mostrarémos la declaración.
\bverse El rey respondió, y dijo a los Caldeos: El negocio se me fué de la memoria: si no me mostráis el sueño y su declaración, seréis hechos cuartos, y vuestras casas serán puestas por muladares.
\bverse Y si mostrareis el sueño y su declaración, recibiréis de mí dones, y mercedes, y grande honra: por tanto mostrádme el sueño, y su declaración.
\bverse Respondieron la segunda vez, y dijeron: Diga el rey el sueño a sus siervos, y mostrarémos su declaración.
\bverse El rey respondió, y dijo: Yo conozco ciertamente que vosotros ponéis dilaciones, porque veis que el negocio se me ha ido de la memoria.
\bverse Si no me mostráis el sueño, una sola sentencia será de vosotros. Ciertamente respuesta mentirosa y perversa que decir delante de mí aparejáis vosotros, entre tanto que se muda el tiempo: por tanto decídme el sueño, para que yo entienda que me podéis mostrar su declaración.
\bverse Los Caldeos respondieron delante del rey, y dijeron: No hay hombre sobre la tierra que pueda declarar el negocio del rey: además de esto, ningún rey, príncipe, ni señor preguntó cosa semejante a ningún mago, ni astrólogo, ni Caldeo.
\bverse Finalmente el negocio que el rey demanda es singular, ni hay quien lo pueda declarar delante del rey, salvo los dioses, cuya morada no es con la carne.
\bverse Por esto el rey con ira y con grande enojo mandó que matasen a todos los sabios de Babilonia.
\bverse Y el mandamiento se publicó, y los sabios eran llevados a la muerte; y buscaron a Daniel, y a sus compañeros para matarlos.
\bverse Entónces Daniel habló avisada y prudentemente a Arioc, capitán de los de la guarda del rey, que había salido para matar los sabios de Babilonia.
\bverse Habló, y dijo a Arioc, capitán del rey: ¿Qué es la causa que este mandamiento se publica de parte del rey tan apresuradamente? Entónces Arioc declaró el negocio a Daniel.
\bverse Y Daniel entró, y pidió al rey que le diese tiempo, y que él mostraría al rey la declaración.
\bverse Entónces Daniel se fué a su casa; y declaró el negocio a Ananías, Misael, y Azarías sus compañeros;
\bverse Para demandar misericordias del Dios del cielo sobre este misterio, y que Daniel y sus compañeros no pereciesen con los otros sabios de Babilonia.
\bverse Entónces el misterio fué revelado a Daniel en visión de noche: por lo cual Daniel bendijo al Dios del cielo;
\bverse Y Daniel habló, y dijo: Sea bendito el nombre de Dios de siglo hasta siglo; porque suya es la sabiduría y la fortaleza.
\bverse Y él es el que muda los tiempos, y las oportunidades: quita reyes, y pone reyes: da la sabiduría a los sabios, y la ciencia a los entendidos:
\bverse El revela lo profundo y lo escondido: conoce lo que está en tinieblas, y la luz mora con él.
\bverse A tí, o! Dios de mis padres, te doy las gracias, y te alabo, que me diste sabiduría y fortaleza; y ahora me enseñaste lo que te pedímos, porque nos ensenaste el negocio del rey.
\bverse Después de esto Daniel entró a Arioc, al cual el rey había puesto para matar a los sabios de Babilonia: fué y díjole así: No mates los sabios de Babilonia: méteme delante del rey, que yo mostraré al rey la declaración.
\bverse Entónces Arioc metió prestamente a Daniel delante del rey, y díjole así: Un varón de los trasportados de Judá he hallado, el cual declarará al rey la interpretación.
\bverse Respondió el rey, y dijo a Daniel, (al cual llamaban Baltasar:) ¿Podrás tú hacerme entender el sueño que ví, y su declaración?
\bverse Daniel respondió delante del rey, y dijo: El misterio que el rey demanda, ni sabios, ni astrólogos, ni magos, ni adivinos lo pueden enseñar al rey.
\bverse Mas hay un Dios en los cielos el cual revela los misterios; y él ha hecho saber al rey Nabucodonosor lo que ha de acontecer a cabo de dias. Tu sueño, y las visiones de tu cabeza sobre tu cama, es esto:
\bverse Tú, o! rey, en tu cama, tus pensamientos subieron por saber lo que había de ser en lo porvenir; y el que revela los misterios, te mostró lo que ha de ser.
\bverse Y a mí, no por la sabiduría que en mí hay más que en todos los vivientes, ha sido revelado este misterio, mas para que yo notifique al rey la declaración, y que entendieses los pensamientos de tu corazón.
\bverse Tú, o! rey, veías, y he aquí una grande imágen. Esta imágen, que era muy grande, y cuya gloria era muy sublime, estaba en pié delante de tí, y su vista era terrible.
\bverse La cabeza de esta imágen era de fino oro: sus pechos y sus brazos de plata: su vientre y sus muslos de metal:
\bverse Sus piernas de hierro: sus piés en parte de hierro, y en parte de barro cocido.
\bverse Estabas mirando, hasta que una piedra fué cortada, no con manos, la cual hirió a la imágen en sus piés de hierro y de barro cocido, y los desmenuzó.
\bverse Entónces fué también desmenuzado el hierro, el barro cocido, el metal, la plata, y el oro, y se tornaron como tamo de las eras del verano; y levantólos el viento, y nunca más se les halló lugar. Mas la piedra que hirió a la imágen, fué hecha un gran monte, que hinchió toda la tierra.
\bverse Este es el sueño: la declaración de él diremos también en la presencia del rey.
\bverse Tú, o! rey, eres rey de reyes; porque el Dios del cielo te ha dado el reino, la potencia, y la fortaleza, y la majestad.
\bverse Y todo lo que habitan hijos de hombres, bestias del campo, y aves del cielo, ha entregado en tu mano; y te ha hecho enseñorear sobre todo ello: tú eres aquella cabeza de oro.
\bverse Y después de tí se levantará otro reino menor que tú; y otro tercero reino de metal, el cual se enseñoreará de toda la tierra.
\bverse Y el reino cuarto será fuerte como hierro; y como el hierro desmenuza, y doma todas las cosas, y como el hierro que quebranta todas estas cosas, desmenuzará y quebrantará.
\bverse Y lo que viste los piés y los dedos en parte de barro cocido de ollero, y en parte de hierro, el reino será diviso, y habrá en él algo de fortaleza de hierro, de la manera que viste el hierro mezclado con el tiesto de barro.
\bverse Y los dedos de los piés en parte de hierro, y en parte de barro cocido, en parte el reino será fuerte, y en parte será frágil.
\bverse Cuanto a lo que viste el hierro mezclado con tiesto de barro, mezclarse han con simiente humana: mas no se pegarán el uno con el otro, como el hierro no se mezcla con el tiesto.
\bverse Mas en los dias de estos reyes el Dios del cielo levantará un reino que eternalmente no se corromperá; y este reino no será dejado a otro pueblo: el cual desmenuzará, y consumirá todos estos reinos, y él permanecerá para siempre.
\bverse De la manera que viste que del monte fué cortada una piedra, que no con manos, desmenuzó al hierro, al metal, al tiesto, a la plata, y al oro, el Dios grande mostró al rey lo que ha de acontecer en lo porvenir. Y el sueño es verdadero, y fiel su declaración.
\bverse Entónces el rey Nabucodonosor cayó sobre su rostro, y humillóse a Daniel, y mandó que le sacrificasen presentes y perfumes.
\bverse El rey habló a Daniel, y dijo: Ciertamente que el Dios vuestro es Dios de dioses, y el Señor de los reyes, y el descubridor de los misterios, pues pudiste revelar este misterio.
\bverse Entónces el rey magnificó a Daniel, y le dió muchos y grandes dones, y púsole por gobernador de toda la provincia de Babilonia, y por príncipe de los gobernadores sobre todos los sabios de Babilonia.
\bverse Y Daniel demandó del rey, y él puso sobre los negocios de la provincia de Babilonia a Sidrac, Misac, y Abdenago: y Daniel a la puerta del rey.


