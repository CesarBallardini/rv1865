% Capítulo 2
\bchapter
%\lehead[\hspace{-3em}\Large\pagemark\quad\normalfont\headerfont Creation de l'homme.\quad Genese.]{}
\pagestyle{scrplain}

% pp. 750  https://books.google.cl/books?id=GMDUAAAAMAAJ&hl=es&pg=PA749#v=onepage&q&f=false

\begin{chaptercomment}
Habiendo soñado Nabuchodonosor un sueño divino, y
  habiéndosele olvidado, y no habiendo en Babylonia
  sábio que se lo pudiese acordar para declararselo,
  Daniel se presenta, y le reduce á la memoria por
  revelacion de Dios, no solo el sueño, mas aun las
  ocasiones  de él.  II. La declaracion del sueño era, que
  en figura de una estatua de diversas materias, le
  pinta Dios tres monarquías, que habian de suceder 
  despues de la de los Chaldeos (á saber, la de los Persas,
  la de los Griegos, y la de los Romanos) y sus fortunas;
  y que en el progreso de la cuarta apareceria
  el reino de Cristo glorioso, que naciendo de muy
  bajo principio, y sin ninhuna fuerza ni apariencia 
  humana, abatiria toda la gloria del mundo, y creceria
  en inmensa y eterna gloria.
\end{chaptercomment}

\vspace{\baselineskip}

\bversenonum \lettrine[lines=3,loversize=-0.2,lraise=0.2]{Y}{en} el segundo año del reino de
  Nabuchodonosor, soñó Nabuchodonosor sueños, y su espíritu se
  quebrantó, y su sueño se huyó de él.

\bverse Y mandó el rey llamar magos, astrólogos,
  y encantadores, y Chaldeos, para 
  que enseñasen al rey sus sueños: los
  cuales vinieron, y se presentaron delante
  del rey.
\bverse Y el rey les dijo: He soñado \emph{un} sueño, 
  y mi espíritu se ha quebrantado por
  saber el sueño.
\bverse Y los Chaldeos hablaron al rey en
  Syriaco: Rey, para siempre vive: \emph{Dí el}
  sueño á tus siervos, y mostrarémos la
  declaración.
\bverse El rey respondió, y dijo á los Chaldeos:
  El negocio se me fué \emph{de la memoria}:
  si no me mostrais el sueño y su
  declaracion, sereis hechos cuartos, y 
  vuestras casas serán puestas por muladares.
\bverse Y si mostrareis el sueño y su 
  declaracion, recibireis de mí dones, y mercedes,
  y grande honra: por tanto mostrádme
  el sueño, y su declaracion.
\bverse Respondieron la segunda vez, y dijeron:
  Diga el rey el sueño á sus siervos,
  y mostrarémos su declaración.
\bverse El rey respondió, y dijo: Yo conozco
  ciertamente que vosotros ponéis dilaciones,
  porque veis que el negocio se
  me ha ido \emph{de la memoria}.
\bverse Si no me mostráis el sueño, una sola
  sentencia será de vosotros. Ciertamente 
  respuesta mentirosa y perversa que decir
  delante de mí aparejais vosotros,
  entre tanto que se muda el tiempo: por 
  tanto decídme el sueño, para que \emph{yo}
  entienda que me podeis mostrar su
  declaracion.
\bverse Los Chaldeos respondieron delante
  del rey, y dijeron: No hay hombre sobre 
  la tierra que pueda declarar el negocio
  del rey: además de esto, ningún rey,
  príncipe, ni señor preguntó cosa semejante
  a ningún mago, ni astrólogo, ni Chaldeo.
\bverse Finalmente el negocio que el rey
  demanda es singular, ni hay quien lo
  pueda declarar delante del rey, salvo los
  dioses, cuya morada no es con la carne.
\bverse Por esto el rey con ira y con grande
  enojo mandó que matasen á todos los
  sábios de Babylonia.
\bverse Y el mandamiento se publicó, y los
  sábios eran llevados a la muerte;
  y buscaron á Daniel, y á sus compañeros para
  matarlos.
\bverse Entonces Daniel habló avisada y
  prudentemente a Arioch, capitan de los de
  la guarda del rey, que habia salido para
  matar los sábios de Babylonia.
\bverse Habló, y dijo a Arioc, capitan del
  rey: ¿Qué es la causa que este
  mandamiento se publica de parte del rey tan
  apresuradamente? Entonces Arioc
  declaró el negocio a Daniel.
\bverse Y Daniel entró, y pidió al rey que le
  diese tiempo, y que él mostraria al rey
  la declaracion.
\bverse Entonces Daniel se fué á su casa; y
  declaró el negocio á Ananias, Misael, y
  Azarias sus compañeros;
\bverse Para demandar misericordias del
  Dios del cielo sobre este misterio, y que
  Daniel y sus compañeros no pereciesen
  con los otros sábios de Babylonia.
\bverse Entonces el misterio fué revelado á 
  Daniel en vision de noche: por lo cual
  Daniel bendijo al Dios del cielo;
\bverse Y Daniel habló, y dijo: Sea bendito
  el nombre de Dios de siglo hasta siglo;
  porque suya es la sabiduría y la fortaleza.
\bverse Y él es el que muda los tiempos, y
  las oportunidades: quita reyes, y pone
  reyes: da la sabiduría á los sábios, y la 
  ciencia a los entendidos:
\bverse El revela lo profundo y lo escondido:
  conoce lo que está en tinieblas, y la
  luz mora con él.
\bverse A tí, \emph{o!} Dios de mis padres, te doy
  las gracias, y te alabo, que me diste
  sabiduría y fortaleza; y ahora me enseñaste
  lo que te pedimos, porque nos enseñaste
  el negocio del rey.
\bverse Después de esto Daniel entró á Arioch,
  al cual el rey habia puesto para
  matar á los sábios de Babylonia: fué y
  díjole así: No mates los sábios de 
  Babylonia: méteme delante del rey, que \emph{yo} 
  mostraré al rey la declaracion.
\bverse Entonces Arioch metió prestamente
  a Daniel delante del rey, y díjole así:
  Un varon de los trasportados de Juda he
  hallado, el cual declarará al rey la
  interpretación.
\bverse Respondió el rey, y dijo á Daniel,
  (al cual llamaban Balthasar:) ¿Podrás tú 
  hacerme entender el sueño que ví,
  y su declaración?
\bverse Daniel respondió delante del rey, y
  dijo: El misterio que el rey demanda, ni
  sábios, ni astrólogos, ni magos, ni
  adivinos lo pueden enseñar al rey.
\bverse Mas hay \emph{un} Dios en los cielos el cual
  revela los misterios; y él ha hecho saber
  al rey Nabuchodonosor lo que ha de 
  acontecer a cabo de dias. Tu sueño, y 
  las visiones de tu cabeza sobre tu cama,
  es esto:
\bverse Tú, o! rey, en tu cama, tus
  pensamientos subieron por saber lo que había
  de ser en lo porvenir; y el que revela
  los misterios, te mostró lo que ha de ser.
\bverse Y á mí, no por la sabiduría que en
  mí hay más que en todos los vivientes,
  ha sido revelado este misterio, mas para
  que \emph{yo} notifique al rey la declaración, y
  que entendieses los pensamientos de tu
  corazón.
\bverse Tú, o! rey, veías, y he aquí una grande imágen.
  Esta imágen, que era muy 
  grande, y cuya gloria era muy sublime,
  estaba en pié delante de tí, y su vista \emph{era}
  terrible.
\bverse La cabeza de esta imágen era de fino oro:
  sus pechos y sus brazos de plata:
  su vientre y sus muslos de metal:
\bverse Sus piernas de hierro: sus piés en
  parte de hierro, y en parte de barro cocido.
\bverse Estabas mirando, hasta que una piedra
  fué cortada, no con manos, la cual
  hirió á la imágen en sus piés de hierro
  y de barro cocido, y los desmenuzó.
\bverse Entonces fué también desmenuzado
  el hierro, el barro cocido, el metal, la 
  plata, y el oro, y se tornaron como tamo
  de las eras del verano; y levantólos el
  viento, y nunca más se les halló lugar. 
  Mas la piedra que hirió á la imágen, fué 
  hecha un gran monte, que hinchió toda
  la tierra.
\bverse ¶ Este \emph{es} el sueño: la declaración
  de él diremos también en la presencia
  del rey.
\bverse Tú, o! rey, eres rey de reyes;
  porque el Dios del cielo te ha dado el reino,
  la potencia, y la fortaleza, y la magestad.
\bverse Y todo lo que habitan hijos de hombres,
  bestias del campo, y aves del cielo,
  ha entregado en tu mano; y te ha hecho
  enseñorear sobre todo ello: tú \emph{eres}
  aquella cabeza de oro.
\bverse Y después de tí se levantará otro 
  reino menor que tú; y otro tercero reino
  de metal, el cual se enseñoreará de
  toda la tierra.
\bverse Y el reino cuarto será fuerte como
  hierro; y como el hierro desmenuza, y 
  doma todas las cosas, y como el hierro
  que quebranta todas estas cosas,
  desmenuzará y quebrantará.
\bverse Y lo que viste los piés y los dedos
  en parte de barro cocido de ollero, y en
  parte de hierro, el reino será diviso, y
  habrá en él \emph{algo} de fortaleza de hierro, 
  de la manera que viste el hierro mezclado 
  con el tiesto de barro.
\bverse Y los dedos de los piés en parte de 
  hierro, y en parte de barro cocido, en parte 
  el reino será fuerte, y en parte será frágil.
\bverse Cuanto á lo que viste el hierro mezclado 
  con tiesto de barro, mezclarse han
  con simiente humana: mas no se pegarán
  el uno con el otro, como el hierro 
  no se mezcla con el tiesto.
\bverse Mas en los dias de estos reyes el
  Dios del cielo levantará un reino que
  eternalmente no se corromperá; y este 
  reino no será dejado a otro pueblo: \emph{el cual}
  desmenuzará, y consumirá todos estos 
  reinos, y él permanecerá para siempre.
\bverse De la manera que viste que del monte 
  fué cortada una piedra, que no con 
  manos, desmenuzó al hierro, al metal, al
  tiesto, á la plata, y al oro, el Dios grande
  mostró al rey lo que ha de acontecer en 
  lo porvenir. Y el sueño es verdadero,
  y fiel su declaración.
\bverse Entonces el rey Nabuchodonosor 
  cayó sobre su rostro, y humillóse a Daniel,
  y mandó que le sacrificasen presentes
  y perfumes.
\bverse El rey habló a Daniel, y dijo: 
  Ciertamente que el Dios vuestro es Dios de
  dioses, y el Señor de los reyes, y el 
  descubridor de los misterios, pues pudiste 
  revelar este misterio.
\bverse Entonces el rey magnificó a Daniel, 
  y le dió muchos y grandes dones, y 
  púsole por gobernador de toda la provincia 
  de Babylonia, y por príncipe de los 
  gobernadores sobre todos los sábios
  de Babylonia.
\bverse Y Daniel demandó del rey, y él puso 
  sobre los negocios de la provincia de 
  Babylonia á Sidrach, Misacih, y Abdenago:
  y Daniel a la puerta del rey.

