\bchapter[capítulo]
\lehead[\hspace{-3em}\Large\pagemark\quad\normalfont\headerfont Daniel.]{}

\begin{chaptercomment}
 \footnotemarkchapter{}
Daniel y sus compañeros, siendo escogidos de entre los cautivos de Jerusalem, son
  criados y enseñados liberalmente para el servicio del rey de Babylonia; y guardándose
  de contaminarse en las viandas contra la ley, Dios les da sabiduría
  y gracia delante del rey, mas que á ninguno de todos sus sábios, especialmente
  á Daniel, y se quedan en su servicio.
\end{chaptercomment}

\vspace{\baselineskip}

% pp.749 https://books.google.cl/books?id=GMDUAAAAMAAJ&hl=es&pg=PA749#v=onepage&q&f=false
% ¶

% TBD: span lettrine on 10 lines
\bversenonum \lettrine[lines=6,loversize=0.05,lraise=+0.08]{E}{n}%
\, el año tercero del reino de Joacim, rey de Juda, vino Nabuchodonosor, rey de Babylonia, á Jerusalem, y cercóla.
\bverse Y el Señor entregó en sus manos á Joacim, rey de Juda, y parte de los vasos de la casa de Dios, y trújolos a tierra de Sennaar á la casa de su dios; y metió los vasos en la casa del tesoro de su dios.
\bverse Y dijo el rey á Aspenez príncipe de sus eunucos, que trujese de los hijos de Israel, del linage real, y de los príncipes;
\bverse Muchachos en quien no hubiese alguna mácula, y de buen parecer, y enseñados en toda sabiduría, y sábios en ciencia, y de buen entendimiento, y que tuviesen fuerzas para estar en el palacio del rey, y que les enseñase las letras y la lengua de los Chaldeos.
\bverse Y señalóles el rey racion para cada día, de la racion de la comida del rey, y del vino de su beber; que los criase tres años, para que al fin de ellos estuviesen delante del rey.
\bverse Y fueron entre ellos de los hijos de Juda, Daniel, Ananias, Misael, y Azarias:
\bverse A los cuales el príncipe de los eunucos puso nombres. Y puso á Daniel, Balthasar; y a Ananias, Sidrach; y a Misael, Misach; y a Azarias, Abdenago.
\bverse Y Daniel propuso en su corazón de no contaminarse en la racion de la comida del rey, y en el vino de su beber; y pidió al príncipe de los eunucos de no se contaminar.
\bverse (Y puso Dios a Daniel en gracia, y en buena voluntad con el príncipe de los eunucos.)
\bverse Y dijo el príncipe de los eunucos a Daniel: Tengo temor de mi señor el rey, que señaló vuestra comida, y vuestra bebida: el cual porque verá vuestros rostros mas tristes que los de los muchachos que son semejantes á vosotros, condenaréis para con el rey mi cabeza.
\bverse Y Daniel dijo a Malasar, que era señalado por el príncipe de los eunucos sobre Daniel, Ananias, Misael, y Azarias:
\bverse Prueba, \emph{yo te ruego}, tus siervos diez dias, y dénnos de las legumbres á comer, y agua á beber:
\bverse Y parezcan delante de tí nuestros rostros, y los rostros de los muchachos que comen de la racion de la comida del rey, y segun que vieres, harás con tus siervos.
\bverse Consintió pues con ellos en esto, y probó con ellos diez dias.
\bverse Y al cabo de los diez dias pareció el rostro de ellos mejor, y mas gordo de carne que los otros muchachos, que comian de la racion de la comida del rey.
\bverse Y fué, que Malasar tomaba la racion de la comida de ellos, y el vino de su beber, y dábales legumbres.
\bverse Y a estos cuatro muchachos dióles Dios conocimiento, y inteligencia en todas letras y ciencia: mas Daniel tuvo entendimiento en toda visión y sueños.
\bverse Pasados pues los dias al fin de los cuales dijo el rey que los trujesen, el príncipe de los eunucos los trujo delante de Nabuchodonosor.
\bverse Y el rey habló con ellos, y no fué hallado entre todos ellos otro como Daniel, Ananias, Misael, y Azarias; y estuvieron delante del rey.
\bverse Y en todo negocio de sabiduría e inteligencia que el rey les demandó, los halló diez veces sobre todos los magos y astrólogos que había en todo su reino.
\bverse Y fué Daniel hasta el año primero del rey Cyro.

%% Notes for the page
%\marginnotes{-6.0in}{
%   \fakenote{I}{Ce premier cha- \lb pitre est fort diffi- \lb
% cile~: \& pour cette \lb
% cause, il estoit de- \lb
% fendu entre les He \lb
% brieux de le lire \& \lb
% interpreter devant \lb
% l'aage de trente \lb ans.}
%   \fakenote{a}{Fit de rien, \& \lb sans aucune ma- tiere.}
%   \fakenote{1}{\bibleverse{Job}(38:4), \bibleverse{Ps} \lb
%   33.6, \textit\& 89.12.,\lb 135.5, \bibleverse{Ec} \lb
%   13.1, \bibleverse{Ac} 14-15, \lb
%   \textit\& 17.14}
%}
%% \pagebreak
%
%%skip j counter
%% TBD: do it better
%\addtocounter{footnotemain}{1}
%
%% Notes for the page
%\marginnotes{-16.3cm}{
% % Continue note f from left column
%}
%
%% Page 2
%%\pagebreak
%
%% Notes de gauche
%\marginnotes{-23.5cm}{
%}
%
%% Page 2: deuxieme colonne
%%\vfill\break
%\noindent
%%%%
