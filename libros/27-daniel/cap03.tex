
% pp. 752  https://books.google.cl/books?id=GMDUAAAAMAAJ&hl=es&pg=PA749#v=onepage&q&f=false


3:1 El rey Nabucodonosor hizo una estatua de oro, la altura de la cual era de sesenta codos, su anchura de seis codos: levantóla en el campo de Dura, en la provincia de Babilonia.
3:2 Y envió el rey Nabucodonosor a juntar los grandes, los asistentes y capitanes: oidores, receptores, los del consejo, presidentes, y a todos los gobernadores de las provincias, para que viniesen a la dedicación de la estatua, que el rey Nabucodonosor había levantado.
3:3 Y fueron congregados los grandes, los asistentes, y capitanes, los oidores, receptores, los del consejo, los presidentes, y todos los gobernadores de las provincias, a la dedicación de la estatua que el rey Nabucodonosor había levantado; y estaban en pié delante de la estatua que había levantado el rey Nabucodonosor.
3:4 Y el pregonero pregonaba a alta voz: Mándase a vosotros, pueblos, naciones, y lenguajes:
3:5 En oyendo el son de la bocina, del pífano, del atambor, de la arpa, del salterio, de la sinfonía, y de todo instrumento músico, os postraréis, y adoraréis la estatua de oro que el rey Nabucodonosor ha levantado.
3:6 Y cualquiera que no se prostrare, y la adorare, en la misma hora será echado dentro del horno de fuego ardiendo.
3:7 Por lo cual en oyendo todos los pueblos el son de la bocina, del pífano, del atambor, de la arpa, del salterio, de la sinfonía, y de todo instrumento músico, todos los pueblos, naciones, y lenguajes se postraron, y adoraron la estatua de oro que el rey Nabucodonosor había levantado.
3:8 Por esto en el mismo tiempo algunos varones Caldeos se llegaron, y denunciaron de los Judíos:
3:9 Hablando, y diciendo al rey Nabucodonosor: Rey, para siempre vive.
3:10 Tú, o! rey, pusiste ley, que todo hombre en oyendo el son de la bocina, del pífano, del atambor, de la arpa, del salterio, de la sinfonía, y de todo instrumento músico, se postrase y adorase la estatua de oro:
3:11 Y el que no se postrase, y la adorase, fuese echado dentro del horno de fuego ardiendo.
3:12 Hay unos varones Judíos, los cuales tú pusiste sobre los negocios de la provincia de Babilonia, Sidrac, Misac, y Abdenago: estos varones, o! rey, no han hecho cuenta de tí: no adoran tus dioses, no adoran la estatua de oro, que tú levantaste.
3:13 Entónces Nabucodonosor dijo con ira y con enojo, que trajesen a Sidrac, Misac, y Abdenago: luego estos varones fueron traidos delante del rey.
3:14 Habló Nabucodonosor, y díjoles: ¿Es verdad, Sidrac, Misac, y Abdenago, que vosotros no honráis a mi dios, ni adoráis la estatua de oro que yo levanté?
3:15 Ahora pues, ¿estáis prestos para que en oyendo el son de la bocina, del pífano, del atambor, de la arpa, del salterio, de la sinfonía, y de todo instrumento músico, os postréis, y adoréis la estatua que yo hice? Porque si no la adorareis, en la misma hora seréis echados en medio del horno de fuego ardiendo: ¿Y qué dios será aquel que os libre de mis manos?
3:16 Sidrac, Misac, y Abdenago respondieron, y dijeron al rey Nabucodonosor: No curamos de responderte sobre este negocio.
3:17 He aquí nuestro Dios, a quien honramos, puede librarnos del horno de fuego ardiendo; y de tu mano, o! rey, nos librará.
3:18 Y si no: sepas, o! rey, que tu dios no adorarémos, y la estatua que tú levantaste no honrarémos.
3:19 Entónces Nabucodonosor fué lleno de ira, y la figura de su rostro se demudó sobre Sidrac, Misac, y Abdenago: habló, y mandó que el horno se encendiese siete veces tanto de lo que cada vez solía.
3:20 Y mandó a hombres valientes en fuerza que estaban en su ejército, que atasen a Sidrac, Misac, y Abdenago, para echarlos en el horno de fuego ardiendo.
3:21 Entónces estos varones fueron atados con sus mantos, y sus calzas, y sus turbantes, y sus vestidos, y fueron echados dentro del horno de fuego ardiendo.
3:22 Porque la palabra del rey daba priesa, y había procurado que se encendiese mucho. La llama del fuego mató a aquellos hombres que habían alzado a Sidrac, Misac, y Abdenago.
3:23 Y estos tres varones Sidrac, Misac, y Abdenago cayeron atados dentro del horno de fuego ardiendo.
3:24 Entónces el rey Nabucodonosor se espantó, y se levantó apriesa, y habló, y dijo a los de su consejo: ¿No echamos tres varones atados dentro del fuego? Ellos respondieron, y dijeron al rey: Es verdad, o! rey.
3:25 Respondió, y dijo: He aquí que yo veo cuatro varones sueltos, que se pasean en medio del fuego; y ningún daño hay en ellos; y el parecer del cuarto es semejante a hijo de Dios.
3:26 Entónces allegóse Nabucodonosor a la puerta del horno de fuego ardiendo, y habló, y dijo: Sidrac, Misac, y Abdenago, siervos del Alto Dios, salíd, y veníd. Entónces Sidrac, Misac, y Abdenago salieron de en medio del fuego.
3:27 Y juntáronse los grandes, los gobernadores, y los capitanes, y los del consejo del rey para mirar estos varones, como el fuego no se enseñoreó de sus cuerpos: ni cabello de sus cabezas fué quemado, ni sus ropas se mudaron, ni olor de fuego pasó por ellos.
3:28 Nabucodonosor habló, y dijo: Bendito el Dios de ellos, de Sidrac, Misac, y Abdenago, que envió su ángel, y libró sus siervos que esperaron en él, y el mandamiento del rey mudaron, y entregaron sus cuerpos ántes que sirviesen ni adorasen otro dios que su Dios.
3:29 Por mí pues se pone decreto, que todo pueblo, nación, o lenguaje que dijere blasfemia contra el Dios de Sidrac, Misac, y Abdenago, sea descuartizado, y su casa sea puesta por muladar; por cuanto no hay Dios que pueda librar como este.
3:30 Entónces el rey ennobleció a Sidrac, Misac, y Abdenago en la provincia de Babilonia.

